\documentclass{article}
\usepackage{amsmath,amssymb}
\usepackage{mathtools}
\usepackage{amsfonts}
\usepackage{amssymb}
\usepackage{tikz}

\DeclarePairedDelimiter{\ceil}{\lceil}{\rceil}
\newcounter{question}
\setcounter{question}{0}
\begin{document}

\newcommand\Que[1]{%
   \leavevmode\par
   \stepcounter{question}
   \noindent
   \thequestion. Q --- #1\par}

\newcommand\Ans[2][]{%
    \leavevmode\par\noindent
   {\leftskip37pt
    A --- \textbf{#1}#2\par}}

\Que{  
    Can you repaint two six-sided dice such that
    the distribution of getting the results 
    is still the same, i.e.,
    the number of ways for getting $(2, 3, 4, \dots)$ 
    does not change.
}
\Ans{
    Let's start with using the concept of \textbf{Generating Functions}
    for a standard dice roll. In a standard dice there is one way to roll a $1$, 
    one way to roll a $2$, and so on. This fact can be represented
    as a Generating Function, one that generates the sequence of
    all possibilities when rolling a dice as:\\

    $1 \cdot x + 1 \cdot x^2 + 1 \cdot x^3 + 1 \cdot x^4 + 1 \cdot x^5 + 1 \cdot x^6$
    — one way to roll each result. The powers of $x, 1, 2, 3 \dots$
    are a notation to represent the face (result) that is rolled.\\

    According to the similarities between algebra of polynomials 
    and that of the Generating Functions, it follows that the representation
    of rolling two six-faced dice is:\\

    $(x+x^2+x^3+x^4+x^5+x^6)\cdot(x+x^2+x^3+x^4+x^5+x^6)$\\
    
    \begin{equation}
        =(x+x^2+x^3+x^4+x^5+x^6)^2
    \end{equation}\\

    Hence the present question can be restated as if there
    is a way to rewrite the expression $(1)$ as a product of 
    two polynomials $P$ and $Q$, according to the \textbf{rules}:
    \begin{enumerate}
        \item The sum of coefficients of the terms in $P$ and $Q$ is $6$, each.
        \item The exponents of $x$ are all non-zero integers.\\
    \end{enumerate}

    It can bee seen that voilating \#1 above would result in a dice that
    is \textbf{not} 6-faced, and \#2 is necessary to make
    Generating Functions work.\\

    We start by factorizing $(1)$ as
    a product of $P$ and $Q$ as follows:\\

    $
    \begin{aligned}
        PQ\\
        & = x^2(1+x+x^2+x^3+x^4+x^5)^2
            && \text{factoring out $x$}\\
        & = x^2(1+x)^2(1+x^2+x^4)^2
            && \text{(factoring out $1+x$}\\
        &   && \text{since $x=-1$ is a root}\\
        &   && \text{$1 - 1 + 1 -1 + 1 - 1 = 0$,}\\
        &   && \text{and using long-division).}\\
    \end{aligned}
    $\\

    Now to factorize $(1+x^2+x^4)$ we observe that
    it is a quadratic in $x^2$, $[1+(x^2) + (x^2)^2]$.
    To solve the following for $x^2$:\\

    \begin{equation}
        1 + (x^2) + (x^2)^2 = 0
    \end{equation}\\

    $
    x^2 = \frac{-b \pm \sqrt{b^2-4ac}}{2a}$\\

    $
    = \frac{-1 \pm \sqrt{-3}}{2}
    = \frac{-1 \pm \sqrt{3}i}{2}
    = -\frac{1}{2} \pm \frac{\sqrt{3}}{2} i$.\\

    Solving for complex roots,\\

    $
    x = \pm \frac{1}{2} \pm \frac{\sqrt{3}}{2} i
    $ are the 4 roots to (2). Lets call them
    $r_1, \bar{r_1}, r_2, \bar{r_2}$.
    These are \textbf{complex conjugate pairs}.
    The sums \& products of the complex conjugates 
    can be calculated as:\\

    $
    \boxed{
        \begin{aligned}
            r_1+\bar{r_1}\\
            & = (\frac{1}{2} + \frac{\sqrt{3}}{2} i)
            + (\frac{1}{2} - \frac{\sqrt{3}}{2} i)\\
            & = 1\\
            r_1\bar{r_1}\\
            & = (\frac{1}{2} + \frac{\sqrt{3}}{2} i)
            (\frac{1}{2} - \frac{\sqrt{3}}{2} i)\\
            & = \frac{1}{4}-\frac{3}{4}(-1)\\
            & = \frac{1}{4}+\frac{3}{4}\\
            & = 1
        \end{aligned}
        }
    $\\

    Similar results can be found for $r_2$ and $\bar{r_2}$.
    Therefore $(2)$ can be written as:\\

    $
    \begin{aligned}
        1 + x^2 + x^4\\
        & = (x-r_1)(x-\bar{r_1})(x-r_2)(x-\bar{r_2})\\
        & = [x^2 - x(r_1+\bar{r_1}) + r_1\bar{r_1}]
        [x^2 - x(r_2+\bar{r_2}) + r_2\bar{r_2}]\\
        & = (x^2 - x + 1)(x^2 + x + 1)
            && \text{(using properties}\\
        &   && \text{of conjugates)}.
    \end{aligned}
    $\\

    To continue factorizinbg $(1)$:\\

    $
    \begin{aligned}
        (P)(Q)\\
        & = x^2(1+x)^2(1 - x + x^2)^2(1 + x + x^2)^2
            && \text{using (2) above}\\
    \end{aligned}
    $\\

    We want to break this up as a product of $P,Q$
    so they each represent a six-faced dice. The number
    of coefficients carried by each of these terms
    are as follows:\\

    \begin{tabular}{|c|c|c|c|}
    \hline
    $(x)(x)$ & $(1+x)(1+x)$ & $(1 - x + x^2)(1 - x + x^2)$ & $(1 + x + x^2)(1 + x + x^2)$\\
    \hline
    $1+1$ & $2+2$ & $1+1$ & $3+3$ \\
    \hline
    \end{tabular}\\

    Since we want each of $P,Q$ to contain $6$ 
    as a sum of coefficients, we need the terms with
    sums of $2$ and $3$ in each $P$ and $Q$:\\
    
    \begin{tabular}{|c|c|}
    \hline
    $P$ & $Q$ \\
    \hline
    $\left[(1+x)(1+x+x^2)\dots\right]$ 
        & $\left[(1+x)(1+x+x^2)\dots\right]$ \\
    \hline
    \end{tabular}\\

    On expansion in their current form, both $P$ and $Q$
    would generate a \textbf{constant term} (one without a power of x)
    which would amount to a face with no number on it. This is
    against the rules of the desired dice (see above). 
    To account for this this we must use an $x$ term in each $P$ and $Q$:\\
    
    \begin{tabular}{|c|c|}
    \hline
    $P$ & $Q$ \\
    \hline
    $\left[(x)(1+x)(1+x+x^2)\dots\right]$ 
        & $\left[(x)(1+x)(1+x+x^2)\dots\right]$ \\
    \hline
    \end{tabular}\\

    To use the remaining terms we have two choices.
    We could use one in each $P$ and $Q$, but this
    will result in two dice identical to the original ones.
    The other choice is to use them both in either $P$ or $Q$:\\
    
    \begin{tabular}{|c|c|}
    \hline
    $P$ & $Q$ \\
    \hline
    $\left[(x)(1+x)(1+x+x^2)(1-x+x^2)\right]$ 
        & $\left[(x)(1+x)(1+x+x^2)\right]$ \\
    \hline
    \end{tabular}\\

    Upon expansion, the product $PQ$ looks like:\\

    $
    (x^8+x^6+x^5+x^4+x^3+x)
    (x^4+2x^3+2x^2+x)
    $,\\

    both with non-zero coefficients, and the
    sum of coefficients being $6$.\\

    We now have the desired pair of non-standard dice painted:\\

    $[8,6,5,4,3,1]$ and 
    $[4,3,3,2,2,1]$.\\

    Since this pair is derived from the original
    pair of standard dice, the distribution of
    getting the results $2, 3, 4 \dots$ is the same.
    }
\end{document}