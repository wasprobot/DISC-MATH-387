\documentclass{article}
\usepackage{amsmath,amssymb}
\usepackage{mathtools}
\usepackage{amsfonts}
\usepackage{amssymb}
\usepackage{tikz}

\DeclarePairedDelimiter{\ceil}{\lceil}{\rceil}
\newcounter{question}
\setcounter{question}{0}
\begin{document}

\newcommand\Que[1]{%
   \leavevmode\par
   \stepcounter{question}
   \noindent
   \thequestion. Q --- #1\par}

\newcommand\Ans[2][]{%
    \leavevmode\par\noindent
   {\leftskip37pt
    A --- \textbf{#1}#2\par}}

\Que{
    Find the general solution of the recurrence equation\\
    $g_{n+2} = 3g_{n+1}-2g_n$.
}
\Ans{
    The recurrence can be written as a homogenous
    equation, \\
    $g_{n+2} - 3g_{n+1} + 2g_n=0$. On summating:
    \begin{equation}
        \sum_{n=0}^\infty {g_{n+2}} 
        - 3\sum_{n=0}^\infty {g_{n+1}} 
        + 2\sum_{n=0}^\infty {g_n}=0
    \end{equation}

    Let $f(x)$ represent the generating function for the sequence
    $\{r_n\colon n\geq 0\}$:
    \begin{equation*}
        f(x) = \sum_{n=0}^\infty g_n x^n 
        = g_0+g_1x+g_2x^2+g_3x^3+\cdots
    \end{equation*}
    
    Multiplying $(1)$ by $x^n$:
    \begin{equation}
        \sum_{n=0}^\infty {g_{n+2}x^n} 
        - 3\sum_{n=0}^\infty {g_{n+1}x^n} 
        + 2\sum_{n=0}^\infty {g_nx^n} = 0
    \end{equation}
    
    We observe that:
    \begin{align*}
        \sum_{n=0}^\infty {g_{n+2}x^n} & = 
        g_2+g_3x+g_4x^2+g_5x^3+\cdots\\
        x^2\sum_{n=0}^\infty {g_{n+2}x^n} & = 
        g_2x^2+g_3x^3+g_4x^4+g_5x^5+\cdots\\
        & = f(x) - g_0 - g_1x\\
        \therefore
        \sum_{n=0}^\infty {g_{n+2}x^n} & = 
        \frac{f(x) - g_0 - g_1x}{x^2}
    \end{align*}

    Also that:
    \begin{align*}
        \sum_{n=0}^\infty {g_{n+1}x^n} & = 
        g_1+g_2x+g_3x^2+g_4x^3+\cdots\\
        x\sum_{n=0}^\infty {g_{n+1}x^n} & = 
        g_1x+g_2x^2+g_3x^3+g_4x^4+\cdots\\
        & = f(x) - g_0\\
        \therefore
        \sum_{n=0}^\infty {g_{n+1}x^n} & = 
        \frac{f(x) - g_0}{x}
    \end{align*}

    Therefore $(2)$ can be re-written as:
    \begin{equation}
        \frac{f(x) - g_0 - g_1x}{x^2}
        - 3\frac{f(x) - g_0}{x}
        + 2f(x) = 0
    \end{equation}

    Simplifying:
    \begin{equation}
        \frac{
            f(x) - g_0 - g_1x
            - 3(f(x)x^2 - g_0x^2)
            + 2f(x)x^2
        }{x^2}
        = 0
    \end{equation}
    \begin{align*}
        f(x)(1-x^2)
        - g_0 - g_1x
        + 3g_0x^2
        & = 0\\
        f(x)(1-x^2)
        & =
        g_0 + g_1x
        - 3g_0x^2\\
        f(x)
        & =
        \frac{g_0 + g_1x - 3g_0x^2}{1-x^2}\\
        & = \frac{g_0 + g_1x - 3g_0x^2}{(1+x)(1-x)}\\
        & = \frac{A}{1+x} + \frac{B}{1-x}\\
        & = \frac{A(1-x)+B(1+x)}{1-x^2}\\
        & = \frac{(A+B) + x(B-A)}{1-x^2}\\
        \frac{g_0 + g_1x - 3g_0x^2}{1-x^2}
        & = \frac{(A+B) + x(B-A)}{1-x^2}\\
        A+B & = g_0\\
        B-A & = g_1\\
        2B & = g_0 + g_1, B = \frac{g_0+g_1}{2}\\
        \text{and, } 2A & = g_0 - g_1, A = \frac{g_0-g_1}{2}\\
        f(x) & = \frac{g_0-g_1}{2(1+x)} + \frac{g_0+g_1}{2(1-x)}\\
    \end{align*}

    Using the geometric series expansion of $\frac{1}{1-x}$
    and $\frac{1}{1+x}$, we conclude that
    $
        f(x) = \frac{g_0-g_1}{2} \sum{(-1)^nx^n}
        + \frac{g_0+g_1}{2} \sum{x^n}
    $. In other words\\
    
    \boxed{
        r_n = \frac{g_0-g_1}{2} (-1)^n
        + \frac{g_0+g_1}{2}
    }
}
\end{document}
