\documentclass{article}
\usepackage{amsmath,amssymb}
\usepackage{mathtools}
\usepackage{amsfonts}
\usepackage{amssymb}
\usepackage{tikz}

\DeclarePairedDelimiter{\ceil}{\lceil}{\rceil}
\newcounter{question}
\setcounter{question}{0}
\begin{document}

\newcommand\Que[1]{%
   \leavevmode\par
   \stepcounter{question}
   \noindent
   \thequestion. Q --- #1\par}

\newcommand\Ans[2][]{%
    \leavevmode\par\noindent
   {\leftskip37pt
    A --- \textbf{#1}#2\par}}

\Que{
    Solve the advancement operator equation\\
    $(A^2+3 A-10)f=0$ if $f(0)=2$
    and $f(1)=10$.
}
\Ans{
    The original equation can be written as $(A+5)(A-2)f=0$.

    $(A-2)f=0$ can also be written as $Af(n) = 2f(n)$, i.e.,
    $f_{n+1}=2f_n$. A function that, when advanced,
    gives twice the value as its predecessor is $f_1=c_1{2^n}$.
    When we try this solution in our original problem we have that
    $(A+5)(A-2)f_1(n) = (A+5)0 = 0$ hence $f_1$ is a solution
    to the original problem.

    Similarly $(A+5)f=0$ has a solution
    $f_2=c_2(-5)^n$ which is also a solution
    to the original problem.

    To see if combined, $f_1$ and $f_2$ give us
    all the solutions to the advancement operator
    equation, we substitute $f(n) = c_12^n + c_2(-5)^n$.

    \begin{align*}
        (A+5)(A-2)f(n) &= (A+5)(A-2)(c_12^n + c_2(-5)^n)\\
        & = (A+5)(
            c_12^{n+1} + c_2(-5)^{n+1}
            - 2(c_12^n + c_2(-5)^n)
            )\\
        & = (A+5)(
            c_12^{n+1} + c_2(-5)^{n+1}
            - c_12^{n+1} - 2c_2(-5)^n
            )\\
        & = (A+5)(
            -7c_2(-5)^n
            )\\
        & = -7c_2(A+5)((-5)^n)\\
        & = -7c_2(-5(-5)^{n} + 5(-5)^{n})\\
        & = 0\\
        \therefore
        f(n) & = c_12^n + c_2(-5)^n \text{ are the solutions.}\\
        f(0) & = c_1 + c_2 = 2\\
        f(1) & = 2c_1 - 5c_2 = 10\\
        & (c_1 = \frac{20}{7}, c_2 = -\frac{6}{7})\\
        & \boxed{f(n) = \frac{20}{7}2^n - \frac{6}{7}(-5)^n}
    \end{align*}

}
\end{document}
