\documentclass{article}
\usepackage{amsmath,amssymb}
\usepackage{mathtools}
\usepackage{amsfonts}
\usepackage{amssymb}
\usepackage{tikz}

\DeclarePairedDelimiter{\ceil}{\lceil}{\rceil}
\newcounter{question}
\setcounter{question}{0}
\begin{document}

\newcommand\Que[1]{%
   \leavevmode\par
   \stepcounter{question}
   \noindent
   \thequestion. Q --- #1\par}

\newcommand\Ans[2][]{%
    \leavevmode\par\noindent
   {\leftskip37pt
    A --- \textbf{#1}#2\par}}

\Que{  
    Number of binary strings of length $n$ with no isolated ones.
}
\Ans{
    Let $S(n)$ be the desired number.\\

    $S(0) = 1, \{''\}$\\

    $S(1) = 1, \{0\}$\\

    $S(2) = 2, \{00,11\}$\\

    $S(3) = 4, \{000,
    011,110,
    111\}$\\
    
    $S(4) = 7, \{0000,
    0011,0110,1100,
    0111,1110,
    1111\}$\\

    $S(5) = 12, \{00000,
    00011,00110,01100,11000,
    00111,01110,11100,
    \textbf{01111,11110,
    11011,}
    11111\}$\\\\

    \textbf{The strings in bold are courtesy of Cristian!}. 
    And we can establish that:\\

    \boxed{$S(n)=S(n-1)+S(n-2)+S(n-4)$}
}
\end{document}
