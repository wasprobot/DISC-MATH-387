\documentclass{article}
\usepackage{amsmath,amssymb}
\newcounter{question}
\setcounter{question}{0}
\begin{document}

\newcommand\Que[1]{%
   \leavevmode\par
   \stepcounter{question}
   \noindent
   \thequestion. Q --- #1\par}

\newcommand\Ans[2][]{%
    \leavevmode\par\noindent
   {\leftskip37pt
    A --- \textbf{#1}#2\par}}

\Que{
    Sophie's coefficients $G(n,k)$ for $n>=0$, $k$ arbitrary integer are defined as follows:
    \[
    G(n,k)= 
    \begin{cases}
        1, & \text{if } n=k=0\\
        0, & \text{if } k<-n \text{ or } k>n\\
        G(n-1,k-1)+G(n-1,k)+G(n-1,k+1), & \text{for } n>0 \text{ and } -n<=k<=n
    \end{cases}
    \]
    Compute the coefficients $G(4,0)$ and $G(4,1)$.
    }
\Ans{The recursive definition of $G$ is used to calculate the values as follows:
    \begin{equation*}
        \begin{aligned}
            \boldsymbol{G(4,0)} &=  G(3,-1)+G(3,0)+G(3,1)       &=6+6+4=\boldsymbol{16}\\
            \\\hline\\
            G(3,-1) &=  G(2,-2)+G(2,-1)+G(2,0)      &=1+2+3=6\\
            G(3,0)  &=  G(2,-1)+G(2,0)+G(2,1)       &=2+3+1=6\\
            G(3,1)  &=  G(2,0)+G(2,1)+G(2,2)        &=3+1+0=4\\
            \\\hline\\
            G(2,-2) &=  G(1,-3)+G(1,-2)+G(1,-1)     &=1\\
            G(1,-3) &&= 0\\
            G(1,-2) &&= 0\\
            \\\hline\\
            G(1,-1) &=  G(0,-2)+G(0,-1)+G(0,0)      &=1\\
            G(0,-2) &&= 0\\
            G(0,-1) &&= 0\\
            G(0,0)  &&= 1\\
            \\\hline\\
            G(2,-1) &=  G(1,-2)+G(1,-1)+G(1,0)      &=0+1+1=2\\
            G(1,-2) &&= 0\\
            G(1,-1) &&= 1\\
            G(1,0)  &=  G(0,-1)+G(0,0)+G(0,1)       &=0+1+0=1\\
            G(0,1)  &&= 0\\
            \\\hline\\
            G(2,0)  &=  G(1,-1)+G(1,0)+G(1,1)       &=1+1+1=3\\
            G(1,1)  &=  G(0,0)+G(0,1)+G(0,2)        &=1+0+0=1\\
            \\\hline\\
            G(2,1)  &=  G(1,0)+G(1,1)+G(1,2)        &=1+0+0=1\\
            G(1,2)  &&=  0\\
            \\\hline\\
            G(2,2)  &=  G(1,1)+G(1,2)+G(1,3)        &=0+0+0=0\\
            G(1,3)  &&= 0\\
        \end{aligned}
    \end{equation*}
    \begin{equation*}
        \begin{aligned}
            \boldsymbol{G(4,1)} &=  G(3,0)+G(3,1)+G(3,2)       &=6+4+1=\boldsymbol{11}\\
            \\\hline\\
            G(3,2)  &=  G(2,1)+G(2,2)+G(2,3)                    &=1+0+0=1\\
            G(2,3)  &&= 0\\
        \end{aligned}
    \end{equation*}
    }
\end{document}