\documentclass{article}
\usepackage{amsmath,amssymb}
\usepackage{mathtools}
\usepackage{amsfonts}
\usepackage{amssymb}
\usepackage{tikz}

\DeclarePairedDelimiter{\ceil}{\lceil}{\rceil}
\newcounter{question}
\setcounter{question}{0}
\begin{document}

\newcommand\Que[1]{%
   \leavevmode\par
   \stepcounter{question}
   \noindent
   \thequestion. Q --- #1\par}

\newcommand\Ans[2][]{%
    \leavevmode\par\noindent
   {\leftskip37pt
    A --- \textbf{#1}#2\par}}

\Que{
    Solve $ a_{n+1}=2a_{n}-2, n\ge1, a_0=5 $
    using Generating Functions.
    }
\Ans{
    This is a non-homogeneous recurrence function,
    $2a_n - a_{n+1} = 2$. 
    Let $f(x) = a_0 + a_1x + a_2x^2 + a_3x^3 + \dots$ 
    represent the generating function of the desired sequence 
    $\{a_n: n\ge{0}\}$. On multiplying both sides
    of the recurrence by $x^n$ and summating, we obtain

    \begin{equation}
        2\Sigma{a_nx^n} 
        - \Sigma{a_{n+1}x^n} 
        = 2\Sigma{x^n}
    \end{equation}
    
    Since $\Sigma{a_{n+1}x^n}$ 
    has $a_{n+1}$ as the coefficient on $x_n$
    and is missing $a_0$, it can be written as
    $\frac{f(x)-a_0}{x} = \frac{f(x)-5}{x}$, 
    using initial value.
    Thus $(1)$ is re-written as
    
    \begin{equation}
        2f(x)-\frac{f(x)-5}{x}
        =2\Sigma{x^n}
        =2\frac{1}{1-x}
    \end{equation}

    Simplifying:\\

    $\frac{2xf(x)-f(x)+5}{x}
    =\frac{2}{1-x}$\\

    $2xf(x)-f(x)+5
    =\frac{2x}{1-x}$\\

    $f(x)(2x-1)
    =\frac{2x}{1-x} - 5$

    \begin{equation}
        f(x)
        =\frac{5-7x}{(1-2x)(1-x)}
    \end{equation}

    Using partial factorization,
    $ f(x) = \frac{A}{1-2x} + \frac{B}{1-x}
     = \frac{A(1-x) + B(1-2x)}{(1-2x)(1-x)}$.
    I.e.,
    $5-7x = A(1-x) + B(1-2x)$ for all $x$.
    Using $x=1$, $B=2$ and using
    $x=\frac{1}{2}$, $A=3$. Thus $(3)$
    is re-written as

    \begin{align*}
        f(x) & = \frac{3}{1-2x} + \frac{2}{1-x}\\
        & = 3\Sigma{2^nx^n} + 2\Sigma{x^n}
        && \text{(using Geometric Series)}
    \end{align*}

    We now observe that the coefficient
    on $x^n$ in $f(x)$ is:\\

    \boxed{
    a_n = 3\cdot{2^n}+2
    }\\

    which is the desired solution.
    }
\end{document}