\documentclass{article}
\usepackage{amsmath,amssymb}
\usepackage{mathtools}
\usepackage{amsfonts}
\usepackage{amssymb}
\usepackage{tikz}

\DeclarePairedDelimiter{\ceil}{\lceil}{\rceil}
\newcounter{question}
\setcounter{question}{0}
\begin{document}

\newcommand\Que[1]{%
   \leavevmode\par
   \stepcounter{question}
   \noindent
   \thequestion. Q --- #1\par}

\newcommand\Ans[2][]{%
    \leavevmode\par\noindent
   {\leftskip37pt
    A --- \textbf{#1}#2\par}}

\Que{
  Perform the algorithm described in Theorem 5.13 on the graph in 
  Exercise 8. Make sure you do not try to be smarter than the algorithm, 
  in particular the following sentence followed exactly: 
  \textit{"We then choose the least integer $i$ for which there is an edge 
  incident with $x_i$ that has not already been traversed."}
    }
\Ans{
  We start with the shortest circuit starting at $(1)$\\
  \begin{align*}
    C &=(1) && \text{next: the smallest \textit{incident} edge not yet visited: visit $(4)$}\\
    &=(1,4,8,3,2,7,12,2,9,3,5,6,1) \\
    &=(1,4,8,3,9,2,7,12,2,3,5,6,1) \\
    &=(1,4,8,3,9,2,7,12,2,10,9,3,5,6,1) \\
    &=(1,4,8,3,9,2,7,12,2,11,4,10,9,3,5,6,1) \\
    &=(1,4,9,2,3,5,6,1) && \\
    &=(1,4,11,2,7,12,2,3,5,6,1) && \\
    &=(1,4,11,2,7,12,2,3,5,6,1) && \\
    ???
  \end{align*}
  }
\end{document}