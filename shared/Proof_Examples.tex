\documentclass{article}
\usepackage[utf8]{inputenc}
\usepackage{amsthm}

\usepackage{natbib}
\usepackage{graphicx}

% COMMANDS
\newcommand{\basis}{\newline \noindent {\bf Basis:} }
\newcommand{\ih}{\newline \noindent {\bf Induction hypothesis:}  Assume }
\newcommand{\is}{\newline \noindent {\bf Induction:} }

\title{Proof Examples}
\author{Peter Cappello \\ \tt{cappello@cs.ucsb.edu} }
\date{}

\begin{document}

\maketitle

\section*{Introduction}

My hope in producing these proofs is to enable students to learn by example what I would like to see, at least in form.
If you think any proof below is incorrect, or could be simpler, clearer, or better in any other way, please email me.

I also am producing the latex used to produce these proofs as a way of showing by example that latex is a labor-saving device, 
even for mathematical work that is as ephemeral as homework.

\subsection*{Style}

I am encouraging a mathematical style that I think is appropriate for a student taking a {em first} course where mathematical proofs are the coin of the realm:
As the student matures, some elements can be abbreviated or omitted.
For now, err on the side of giving an explanations where none is needed.
Similarly, strive for a form that is clear to another first-year student.
Indeed, asking another student to read your proof is a good source of feedback.

One source of thoughts about mathematical writing style is a document by Goss~\cite{Goss}.
Another is by Gillman~\cite{Gillman}.

Below, I produce a few style notes, certainly not a comprehensive list.
Also, some of these notes run against the grain of convention.
Please use your own judgment.

\begin{enumerate}

    \item
    It generally is considered bad form to mix mathematical symbols with English words 
    (e.g., $\forall n \in N, n^2$ is greater than 0.)
    I disagree.
    Please use mathematical symbols as much as possible;
    This results in fewer characters, enabling the eye to see more content at once.
    I also disagree, and for the same reasons, with the rule that numbers less than 10 should be written out.
    In my opinion, this is like saying, for the number 7, 
    ``Why use a single universally understood symbol, when we can use a sequence of five letters that only English speakers understand?"
\end{enumerate}

%\begin{figure}[h!]
%\centering
%\includegraphics[scale=1.7]{universe.jpg}
%\caption{The Universe}
%\label{threadsVsSync}
%\end{figure}

\section*{Mathematical Induction}
The following proofs are of exercises in Rosen~\cite{RosenText}, chapter 5: Mathematical Induction.

\subsubsection*{Exercise 62}

Show that $n$ lines separate the plane into $(n^2 + n + 2) / 2$ regions, if no 2 of these lines are parallel and no 3 pass through a common point.

\newtheorem*{thm}{Theorem}
\begin{thm}
$\forall n \in {\bf N}$, $n$ lines in the plane, where no lines are parallel and no 3 lines intersect at the same point, 
partition the plane into $(n^2 + n + 2) / 2$ regions.
\end{thm}

\begin{proof}
By induction on $n$, the number of lines.
\basis
$n = 0$: If the plane is partitioned by 0 lines, there is $1 = (1^2 + 1 + 2)/2$ region.
\ih
$n$ lines in the plane, where no lines are parallel and no 3 lines intersect at the same point, 
partition the plane into $(n^2 + n + 2) / 2$ regions.
\is
Let there be $n$ lines in the plane, where no lines are parallel and no 3 lines intersect at the same point.
Without loss of generality, remove any 1 line.
By the inductive hypothesis, the remaining $n$ lines partition the plane into $(n^2 + n + 2)/2$ regions.
When line $n + 1$ is placed so that it is not parallel to any of the other $n$ lines and it intersects each line $l$ at a point where no other lines intersect line $l$,
then $n$ such intersections are formed.
Order these intersection points from left to right and bottom to top among points that have the same $y$ coordinate.
The semi-infinite region to the left and below the $1^{st}$ intersection point, is divided into 2 regions by line $n + 1$.
Similarly, when this line passes into the region to the right of an intersetion point, it subdivides this regtion into 2 regions.
Since there are $n$ intersection points, this increases the number of regions by $n$, plus 1 more for the subdivided semi-infinite region, totalling $n + 1$ additional regions:
\begin{eqnarray}
r( n + 1) & = & r( n ) + n + 1 \\
          & = & \frac{n^2 + n + 2}{2} + n + 1 \\
          & = & \frac{(n + 1)^2 + (n + 1) + 2}{2}
\end{eqnarray}

Above, the inductive hypothesis is used to go from Eqn. (1) to (2).

\end{proof}

\bibliographystyle{plain}
\bibliography{references}

\end{document}
