\documentclass{article}
\usepackage{amsmath,amssymb}
\usepackage{mathtools}
\usepackage{amsfonts}
\usepackage{amssymb}
\DeclarePairedDelimiter{\ceil}{\lceil}{\rceil}
\newcounter{question}
\setcounter{question}{0}
\begin{document}

\newcommand\Que[1]{%
   \leavevmode\par
   \stepcounter{question}
   \noindent
   \thequestion. Q --- #1\par}

\newcommand\Ans[2][]{%
    \leavevmode\par\noindent
   {\leftskip37pt
    A --- \textbf{#1}#2\par}}

\Que{
    Consider a set X of 10 positive integers, none of which is greater than 100. Show that it has two distinct subsets whose elements have the same sum.
    }
\Ans{
    The number of ways to choose $10$ numbers from ${1, 2, 3, ..., 100}$ are $ 100 \choose{10} $. Let's call these 'containers'.\\\\

    From the set of $10$ numbers, the number of subsets $= 2^{10} = 1024 $. Let's call these 'items'.\\\\

    According to the Pigeonhole Principle, there exists at least one 'container' with $\ceil[\Bigg]{\frac{{100 \choose {10}}}{1024}} $ number of items. 
    This number is clearly larger than $ 2 $ so there are at least $ 2 $ distinct subsets whose elements have the same sum.
    }
\end{document}