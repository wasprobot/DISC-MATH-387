\documentclass{article}
\usepackage{amsmath,amssymb}
\usepackage{mathtools}
\usepackage{amsfonts}
\usepackage{amssymb}
\DeclarePairedDelimiter{\ceil}{\lceil}{\rceil}
\newcounter{question}
\setcounter{question}{0}
\begin{document}

\newcommand\Que[1]{%
   \leavevmode\par
   \stepcounter{question}
   \noindent
   \thequestion. Q --- #1\par}

\newcommand\Ans[2][]{%
    \leavevmode\par\noindent
   {\leftskip37pt
    A --- \textbf{#1}#2\par}}

\Que{
    Explain why question \#2 from Section 4.2.1 takes $O(n^3)$ steps, 
    and question \#3 takes $O(2^n)$ steps. This is mentioned in the text 
    (with different wording, because the big-Oh notation is only introduced 
    in the next section), but no detailed explanation is given. 
    Do use the big-Oh notation in your explanation, at least in the conclusions.
    }
\Ans{
    For \#2, the rough algorithm might be:\\

    1) Pick $n_1$: which can be done in $n$ "passes" through the set $S$\\

    2) Pick $n_2$: which can be done in $~n$ "passes" through the rest of the set $S$\\

    3) Pick $n_3$: which can be done in $~n$ "passes" through the rest of the set $S$\\

    Combined, these 3 series of operations can be performed in $~n^3$ steps, hence the $O(n^3)$.\\

    Similarly, since there are $2^{n-1}-1$ \textit{complementary pairs} of sets to test for question \#3,
    A computer will have to loop $2^{n-1}-1$ which is \textit{no more than} $2^n$ times to determine the answer.\\
    }
\end{document}