\documentclass{article}
\usepackage{amsmath,amssymb}
\usepackage{mathtools}
\usepackage{amsfonts}
\usepackage{amssymb}
\usepackage{tikz}

\DeclarePairedDelimiter{\ceil}{\lceil}{\rceil}
\newcounter{question}
\setcounter{question}{14}

\tikzset{
  mynode/.style={fill,circle,inner sep=2pt,outer sep=0pt}
}

\begin{document}

\newcommand\Que[1]{%
   \leavevmode\par
   \stepcounter{question}
   \noindent
   \thequestion. Q) #1\par}

\newcommand\Ans[2][]{%
    \leavevmode\par\noindent
   {\leftskip37pt
    A) \textbf{#1}#2\par}}

\Que{
  Find an interval representation or
  explain why one doesn't exist.
  \begin{figure}
    \centering
    \def\svgwidth{\columnwidth}
    \scalebox{0.3}{\input{image.pdf_tex}}
  \end{figure}
}
\Ans{
  We start with drawing the interval graph starting at
  (4)\\
  
  \begin{tikzpicture}
    \draw[thick] (0,0) -- (2,0) node[pos=0.5,label=above:4]{};
    \draw[thick] (3,0) -- (5,0) node[pos=0.5,label=above:9]{};
    \draw[thick] (6,0) -- (8,0) node[pos=0.5,label=above:3]{};
    \draw[thick] (9,0) -- (11,0) node[pos=0.5,label=above:2]{};
    \draw[thick] (12,0) -- (14,0) node[pos=0.5,label=above:5]{};

    \draw[thick] (2.8,-1) -- (8.5,-1) node[pos=0.5,label=above:8]{};

    \draw[thick] (1.8,-2) -- (5.5,-2) node[pos=0.5,label=above:6]{};
    
    \draw[thick, dashed, red] (6.8,-2) -- (13,-2) node[pos=0.5,label=above:1?]{};
  \end{tikzpicture}\\

  \textbf{Since (6) is less than both (3) and (1) 
  and has no relation with (4), it is not clear
  the relation between (4) and (1)}. Hence an interval graph
  can not be drawn.
}
\end{document}
