\documentclass{article}
\usepackage{amsmath,amssymb}
\usepackage{mathtools}
\usepackage{amsfonts}
\usepackage{amssymb}
\usepackage{tikz}

\DeclarePairedDelimiter{\ceil}{\lceil}{\rceil}
\newcounter{question}
\setcounter{question}{0}
\begin{document}

\newcommand\Que[1]{%
   \leavevmode\par
   \stepcounter{question}
   \noindent
   \thequestion. Q --- #1\par}

\newcommand\Ans[2][]{%
    \leavevmode\par\noindent
   {\leftskip37pt
    A --- \textbf{#1}#2\par}}

\Que{  
  Prove, without resorting to Kuratowski's Theorem,
  that the Petersen's Graph is non-planar.
  }
\Ans{
  The Petersen's Graph has the following parameters:\\

  Number of vertices, $n=10$\\
  
  Number of edges, $m=15$\\

  Also, the faces that exist in the graph are,
  $f_5,f_6,f_7,f_8,f_9$, where $f_x$ is a face formed by an x-cycle.\\

  If we start pairing $(e,F)$ for every edge, $e$ that touches a face, $F$,
  the total number of such pairs,\\

  $p=5f_5+6f_6+7f_7+8f_8+9f_9$\\

  According to Euler's Theorem, if this graph were to be planar,
  the following will hold: $n+f-m=2$ 
  where $f$ is the total number of faces. Given the parameters
  it follows that:\\

  $f=7$\\

  Also, $f=f_5+f_6+f_7+f_8+f_9$, the total number of faces.\\

  $\therefore f_5+f_6+f_7+f_8+f_9=7$\\
  
  By a conservative estimate, $ f_5=3, f_6=f_7=f_8=f_9=1 $\\

  $\therefore p \ge 45$\\

  HELP!!
  }
\end{document}
