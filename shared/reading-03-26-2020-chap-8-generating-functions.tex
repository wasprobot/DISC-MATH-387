\documentclass{article}
\usepackage{amsmath,amssymb}
\usepackage{mathtools}
\usepackage{amsfonts}
\usepackage{amssymb}
\usepackage{tikz}

\DeclarePairedDelimiter{\ceil}{\lceil}{\rceil}
\newcounter{question}
\setcounter{question}{0}
\begin{document}

\newcommand\Que[1]{%
   \leavevmode\par
   \stepcounter{question}
   \noindent
   \thequestion. Q --- #1\par}

\newcommand\Ans[2][]{%
    \leavevmode\par\noindent
   {\leftskip37pt
    A --- \textbf{#1}#2\par}}

\Que{
    Explain why there is no generating function $ \frac{1}{x} $. 
    In general, try to figure out exactly what the rules are 
    for dividing two generating functions.
    }
\Ans{
    Trying to understand the claim  that $\frac{1}{1-x}$
    is the Generating function $F(x)$ for which $(1-x)F(x)=1$.\\

    Let $G(x) = 1-x = 1 + (-1)x + 0x^2 + \dots$\\
    
    and $F(x) = b_0 + b_1x + b_2x^2 + \dots$\\

    Then according to Proposition $8.3$, $ F(x)G(x) $
    represents the Generating Function on a sequence
    whose $n^{th}$ term is:\\

    $
    c_n = (1)b_n + (-1)b_{n-1} + (0)b_{n-2} + \dots
    $\\

    Therefore $c_1 = (b_1 - b_0)$, 
    $c_2 = (b_2 - b_1)$,
    $c_3 = (b_3 - b_2)\dots$\\

    The Generating Function $ F(x)G(x) $ must be:\\

    $
    c_0 + (b_1 - b_0)x + (b_2 - b_1)x^2 + \dots
    $\\

    \textcolor{red}{It's not clear to me how this $=1$.}\\

    \noindent\rule{11cm}{0.4pt}\\

    This is how I attempted answering the presented question:\\
    
    Let's assume there was a generating function $ F(x) = \frac{1}{x} $.
    By simple arithmeric, it would follow that\\

    $
    xF(x) = 1
    $\\

    By definition of Generating Functions,\\

    $
    F(x) = \sum_{n=0}^{\infty}{a_nx^n}
    = a_0 + a_1x + a_2x^2 + \dots
    $\\

    Therefore by the previous equality,\\

    $
    x \left[ a_0 + a_1x + a_2x^2 + \dots \right] = 1
    $\\

    The two Generating Functions in play here are:\\

    $
    1 = F(x) = (1) + (0)x + (0)x^2 + (0)x^3 + \dots
    $\\

    and
    $
    x = G(x) = (0) + (1)x + (0)x^2 + (0)x^3 + \dots
    $\\

    Therefore according to Proposition $8.3$, the $n^{th}$
    term of the multplication, $ F(x)G(x) $ is:\\

    $
    a_0b_n + a_1b_{n-1} + a_2b_{n-2} + \dots
    $\\

    $
    = (1)b_n + (0)b_{n-1} + (0)b_{n-2} + \dots
    $\\

    \textcolor{red}{And I got lost :)}\\

    \noindent\rule{11cm}{0.4pt}\\

    }
\end{document}