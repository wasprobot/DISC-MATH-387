\documentclass{article}
\usepackage{amsmath,amssymb}
\newcounter{question}
\setcounter{question}{0}
\begin{document}

\newcommand\Que[1]{%
   \leavevmode\par
   \stepcounter{question}
   \noindent
   \thequestion. Q --- #1\par}

\newcommand\Ans[2][]{%
    \leavevmode\par\noindent
   {\leftskip37pt
    A --- \textbf{#1}#2\par}}

\Que{How many strings are there over the alphabet {A,B,C} with exactly 10 A's, 10 B's, and 10 C's? Please write a short explanation.}
\Ans{
    The total number of such strings are:
    \[
        30\choose 10,10,10
    \]
    Since there are 10 A's + 10 B's + 10 C's = 30 total symbols to choose from, 
    but among the A's, the B's and the C's there is no difference in the symbols, the number of rearrangements equals the Multinomial Coefficient:
    \[
        n\choose{k_1,k_2,..,k_r}
    \]
    }
\end{document}