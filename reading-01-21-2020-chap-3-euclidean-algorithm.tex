\documentclass{article}
\usepackage{amsmath,amssymb}
\newcounter{question}
\setcounter{question}{0}
\begin{document}

\newcommand\Que[1]{%
   \leavevmode\par
   \stepcounter{question}
   \noindent
   \thequestion. Q --- #1\par}

\newcommand\Ans[2][]{%
    \leavevmode\par\noindent
   {\leftskip37pt
    A --- \textbf{#1}#2\par}}

\Que{
    Use the \textbf{Euclidean Algorithm} to find integers $a$ and $b$ such that 
    $70a+182b=28$.
    Are the integers $c$ and $d$ such that $70c+182d=30$?
    }
\Ans{
    \begin{center}
    \begin{tabular}{|c c c|} 
    \hline
    Original Pair & Division Expression & GCD \\ [0.5ex] 
    \hline
    $(182, 70)$ & $182=2\cdot{70}+\textbf{42}$ & $(70, 42)$ \\ 
    $(70, 42)$ & $70=1\cdot{42}+\textbf{28}$ & $(42, 28)$ \\ 
    $(42, 28)$ & $42=1\cdot{28}+\textbf{14}$ & $(28, 14)$ \\ 
    $(28, 14)$ & $28=2\cdot{14}+\textbf{0}$ & \textbf{14}  \\ 
    \hline
    \end{tabular}
    \end{center}
    Substituting the GCD ($\textbf{14}$) 
    into the penultimate Division Expression:
    \begin{equation*}
        \begin{aligned}
        14  &=  42 - \textbf{(1)}(28)\\
            &=  42 - \textbf{(1)}(70 - \textbf{(1)}(42))\\
            &=  \textbf{(-1)}(70) + \textbf{(2)}(42)\\
            &=  \textbf{(-1)}(70) + \textbf{(2)}(182 - \textbf{(2)}(70))\\
            &=  \textbf{(-1)}(70) + \textbf{(2)}(182) - \textbf{(4)}(70)\\
            &=  \textbf{(2)}(182) - \textbf{(5)}(70)\\
        \therefore 
        28  &=  \textbf{(4)}(182) - \textbf{(10)}(70)\\
        \end{aligned}
    \end{equation*}
    Since \textbf{30} does not appear in the remainders in the table above,
    it is \textbf{not} possible to attain $70c+182d=30$ with integers, $c$ and $d$.    
    }
\end{document}